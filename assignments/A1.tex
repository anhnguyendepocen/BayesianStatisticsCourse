\documentclass[a4paper,10pt, notitlepage]{report}
\usepackage{geometry}
\geometry{verbose,tmargin=30mm,bmargin=25mm,lmargin=25mm,rmargin=25mm}
\usepackage[utf8]{inputenc}
\usepackage[sectionbib]{natbib}
\usepackage{amssymb}
\usepackage{amsmath}
\usepackage{enumitem}
\usepackage{xcolor}
\usepackage{cancel}
\usepackage{mathtools}
\PassOptionsToPackage{hyphens}{url}\usepackage{hyperref}
\hypersetup{colorlinks=true,citecolor=blue}


\newtheorem{thm}{Theorem}
\newtheorem{lemma}[thm]{Lemma}
\newtheorem{proposition}[thm]{Proposition}
\newtheorem{remark}[thm]{Remark}
\newtheorem{defn}[thm]{Definition}

%%%%%%%%%%%%%%%%%%%% Notation stuff
\newcommand{\pr}{\operatorname{Pr}} %% probability
\newcommand{\vr}{\operatorname{Var}} %% variance
\newcommand{\rs}{X_1, X_2, \ldots, X_n} %%  random sample
\newcommand{\irs}{X_1, X_2, \ldots} %% infinite random sample
\newcommand{\rsd}{x_1, x_2, \ldots, x_n} %%  random sample, realised
\newcommand{\bX}{\boldsymbol{X}} %%  random sample, contracted form (bold)
\newcommand{\bx}{\boldsymbol{x}} %%  random sample, realised, contracted form (bold)
\newcommand{\bT}{\boldsymbol{T}} %%  Statistic, vector form (bold)
\newcommand{\bt}{\boldsymbol{t}} %%  Statistic, realised, vector form (bold)
\newcommand{\emv}{\hat{\theta}}
\DeclarePairedDelimiter\ceil{\lceil}{\rceil}
\DeclarePairedDelimiter\floor{\lfloor}{\rfloor}

% Title Page
\title{Exam 1 (A1)}
\author{Class: Bayesian Statistics \\ Instructor: Luiz Max Carvalho}
\date{26/04/2021}

\begin{document}
\maketitle

\textbf{Turn in date: until 28/04/2021 at 23:59h Brasilia Time.}

\begin{center}
\fbox{\fbox{\parbox{1.0\textwidth}{\textsf{
    \begin{itemize}
    \item Please read through the whole exam before starting to answer;
    \item State and prove all non-trivial mathematical results necessary to substantiate your arguments;
    \item Do not forget to add appropriate scholarly references~\textit{at the end} of the document;
    \item Mathematical expressions also receive punctuation;
    \item You can write your answer to a question as a point-by-point response or in ``essay'' form, your call;
    \item Please hand in a single, \textbf{typeset} ( \LaTeX) PDF file as your final main document.
 Code appendices are welcome,~\textit{in addition} to the main PDF document.
    \item You may consult any sources, provided you cite \textbf{ALL} of your sources (books, papers, blog posts, videos);
    \item You may use symbolic algebra programs such as Sympy or Wolfram Alpha to help you get through the hairier calculations, provided you cite the tools you have used.
    \item The exam is worth $\min\left\{\text{your\:score}, 100\right\}$ marks.
    \end{itemize}}
}}}
\end{center}
% \newpage
% \section*{Hints}
% \begin{itemize}
%  \item a
%  \item b
% \end{itemize}
% 
\newpage

\section*{Background}

This exam covers decision theory, prior elicitation and  point estimation.
You will need a little measure theory here and there and also be sharp with your knowledge of expectations and conditional expectations.

\section*{1. Warm up: stretching our Bayesian muscles with a classic.}

\cite{Bayes1763}\footnote{The paper was published posthumously and read to the Royal Society by Bayes's close friend Richard Price (1723-1791).}: a billiard ball $W$ is rolled on a line of length $L = 1$, with uniform probability of stopping anywhere on $L$. 
It stops at $0 < p < 1$.
A second ball, $O$, is then rolled $n$ times under the same assumptions.
Let $X$ denote the number of times (out of $n$) that $O$ stopped to the left of $W$.
Given $X=x$, \textbf{what inference(s) can we make on $p$}?
You may assume $n\geq2$.

The idea here is to provide a rigorous, principled Bayesian analysis of this $250$ years old problem.
Here are a few road signs to help you in your analysis:
\begin{enumerate}[label=\alph*)]
 \item (10 marks) For the problem at hand, \textbf{carefully} define the parameter space, data-generating mechanism, likelihood function and all of crucial elements of a Bayesian analysis. What is the dominating measure?
 \item (10 marks) Exhibit your posterior measure and sketch its density.
 
 \textit{Hint:} it might be convenient to prove the following proposition:
 \begin{proposition}
 \label{prop:joint_bayes}
 The joint distribution of $p$ and $X$ is given by
 \begin{equation*}
  \pr(a < p < b, X = x) = \int_{a}^{b} \binom{n}{x} t^x(1-t)^{n-x}\,dt.
 \end{equation*}
 \end{proposition}
 \item (10 marks) Provide a Bayes estimator under (i) quadratic and (ii) 0-1 loss\footnote{Recall the discussion in class about how to properly define 0-1 loss as for continuous parameter spaces.};
 \item (10 marks) Contrast the estimators obtained in the previous item  with the maximum likelihood estimator, in terms of (i) Bayes risk and (ii) integrated risk.
 Is any of the estimators preferable according to both risks?
 If not, which estimator should be preferred under each goal risk?
 \item (10 marks) Suppose one observes $X = x = 6$ for $n=9$ rolls.
 Produce an updated prediction of where the next $m$ balls are going to stop along $L$.
 Please provide (i) a point prediction in the form of an expected value and (ii) a full probability distribution.
\end{enumerate}


\section*{2. Proper behaviour.}

In this question we will explore propriety and its implications for inference.
Consider a parameter space $\boldsymbol{\Theta}$ and a sampling model with density $f(x \mid \theta)$.
A density $h : \boldsymbol{\Theta} \to (0, \infty)$ is said to be proper if 
\begin{equation*}
 \int_{\boldsymbol{\Theta}} h(t)\,d\mu(t)  < \infty,
\end{equation*}
where $\mu$ is the dominating measure.
Assuming $\mu$ to be the Lebesgue measure and that the prior $\pi(\theta)$ is proper, show that
\begin{itemize}
 \item[a)] (10 marks) The posterior density,
 \begin{equation*}
  \xi(\theta \mid x) = \frac{f(x\mid \theta)\pi(\theta)}{m(x)},
 \end{equation*}
is proper almost surely.
\item[b)] (10 marks) The Bayes estimator under quadratic loss, $\delta_{\text{S}}(x) = E_\xi[\theta]$ is biased almost surely.
\textit{Hint:} consider what happens to the integrated risk under unbiasedness.
\end{itemize}
Now,
\begin{itemize}
 \item[c)] (10 marks) Take
$$f(x \mid \theta) = \frac{2x}{\theta}, x \in [0, \sqrt{\theta}],$$
and 
$$ \pi(\theta) = \frac{2}{\pi (1 +\theta^2)}, \theta > 0.$$
The posterior density is then
\begin{equation*}
 p(\theta \mid x) = \frac{1}{m(x)} \frac{4x}{\pi}\frac{1}{\theta(1 +\theta^2)},
\end{equation*}
 and thus
\begin{equation*}
 m(x) = \int_{0}^\infty \frac{4x}{\pi} \frac{1}{t(1 +t^2)}\,dt = \frac{4x}{\pi} \int_{0}^\infty \frac{1}{t(1 +t^2)}\,dt = \infty,
\end{equation*}
i.e., the posterior is apparently improper.
Explain how this ``counter-example'' is wrong.
\end{itemize}

\section*{3. Rayleigh dispersion\footnote{This question's title is pun: \url{https://en.wikipedia.org/wiki/Rayleigh_scattering}.}.}

Let $X_1, \ldots, X_n$ be a random sample from a distribution whose probability density function is
\begin{equation*}
 f(x \mid \sigma) = \frac{x}{\sigma^2} \exp\left(-\frac{x^2}{2\sigma^2}\right),\, \sigma>0,\, 0 < x < \infty.
\end{equation*}
This is the Rayleigh\footnote{John William Strutt (1842-1919), the Third Baron Rayleigh, was a British physicist, known for his work on Statistical Mechanics.} distribution, used in reliability modelling, where the $X$ are failure times (time-until-failure).
\begin{itemize}
 \item[a)] (5 marks) For $t$ fixed and known, find $\eta = \pr(X > t)$ as function of $\sigma^2$;
 \item[b)] (5 marks) Find the likelihood function for $\eta$ for a sample $\boldsymbol{x} = \{x_1, \ldots, x_n\}$;
 \item[c)] (10 marks) Is the Rayleigh distribution a member of the exponential family? Justify your answer;
 \item[d)] (10 marks) Find the Jeffreys's prior for $\eta$, $\pi_J(\eta)$;
 \item[e)] (10 marks) Using $\pi_J(\eta)$, derive the predictive distribution $p(\tilde{x} \mid \boldsymbol{x})$.
\end{itemize}

\bibliographystyle{apalike}
\bibliography{a1}
\end{document}          
